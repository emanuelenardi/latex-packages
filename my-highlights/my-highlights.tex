% NOTE: Definizione nuovo ambiente per evidenziare il codice
\makeatletter
	\newenvironment{btHighlight}[1][]
		{\begingroup\tikzset{bt@Highlight@par/.style={#1}}\begin{lrbox}{\@tempboxa}}
		{\end{lrbox}\bt@HL@box[bt@Highlight@par]{\@tempboxa}\endgroup}

	\newcommand\btHL[1][]{%
		\begin{btHighlight}[#1]\bgroup\aftergroup\bt@HL@endenv%
	}
	\def\bt@HL@endenv{%
		\end{btHighlight}%
		\egroup%
	}
	\newcommand{\bt@HL@box}[2][]{%
		\tikz[#1]{%
			\pgfpathrectangle{\pgfpoint{1pt}{0pt}}{\pgfpoint{\wd #2}{\ht #2}}%
			\pgfusepath{use as bounding box}%
			\node[anchor=base west, fill=orange!30,outer sep=0pt,inner xsep=1pt, inner ysep=0pt, rounded corners=3pt, minimum height=\ht\strutbox+1pt,#1]{\raisebox{1pt}{\strut}\strut\usebox{#2}};
		}%
	}
\makeatother

\newcommand{\greenHL}{\btHL[fill=green!30]}
\newcommand{\blueHL}{\btHL[fill=blue!30]}
\newcommand{\azureHL}{\btHL[fill=SkyBlue]}
\newcommand{\redHL}{\btHL}
\newcommand{\yellowHL}{\btHL[fill=yellow!60]}
\newcommand{\orangeHL}{\btHL[fill=orange!60]}
