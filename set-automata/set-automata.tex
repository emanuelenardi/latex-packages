% NOTE: nuovo stato
\newcommand{\newState}[4]{\node[state,#3] (#1)[#4] {#2};}

% NOTE: nuova transizione
\newcommand{\newTransition}[4]{\path[->] (#1) edge [#4] node {#3} (#2);}

% NOTE: per segnalare i responsabili dei follow
\newcommand\blame[2]{%
	% \(\underbrace{#1}_{\mathcolor{blue}{#2}}\)%
	% NOTE: \vphantom{,} è per avere una baseline comune
	% \(\underset{\mathcolor{blue!50!green}{#1}}{\vphantom{,}#2}\)%
	\(\underset{\mathcolor{blue!50!black}{#1}}{\vphantom{,}#2}\)%
}
\setlength\lightrulewidth{0.2pt}

\tikzset{
	node distance = 1cm and 1cm,
	initial distance = 0.5cm,
	initial text = {\emph{inizio}},
	double distance = 1.5pt,
	every state/.style = {
		draw,
		fill = gray!10,
		minimum size = 10mm
	},
	every edge/.style = {
		draw,
		->, >=stealth',
		shorten >= 1pt,
		thin,
		on grid,
	}
}
