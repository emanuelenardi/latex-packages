% WARNING: Mantenee l'ordine dei pacchetti è fonadamentale per non rompere la build
% WARNING: "semantic" da caricare sopo "amsmath"
% IDEA: math suymbols
% NOTE: "amsthm" - teoremi e dimostrazioni
% NOTE: "amsfonts" - nomi insiemi numerici
% NOTE: "amssymb" - leqslant & geqslant
% NOTE: "mathtools" - mathtools = amsmath + other stuff
% NOTE: "MnSymbol" - fornisce le freccie che utilizzo per andare a capo riga nelle liste di codice
% NOTE: "abraces" - angle brakets fine tuning
% NOTE: "braket" - permette l'uso di parentesi angolari
% NOTE: "nicefrac" - divisione in linea
% NOTE: "textgreek" - caratteri greci
% NOTE: "siunitx" - unità di misura del SI
\usepackage{
	,amsthm
	,amsfonts
	,amssymb
	,MnSymbol
	,mathtools
	% ,braket
	,semantic
	,empheq
	,nicefrac
}


% NOTE: comando definito dal package "amsmath"
\DeclarePairedDelimiter\norm{\lVert}{\rVert}
\DeclarePairedDelimiter\abs{\lvert}{\rvert}
\DeclarePairedDelimiter\Abs{\bigg\lvert}{\bigg\rvert}
\DeclarePairedDelimiter\Bracket{\lbrack}{\rbrack}

% NOTE: colorare simboli matematici
% tex.stackexchange.com/questions/21598/
\makeatletter
\def\mathcolor#1#{\@mathcolor{#1}}
\def\@mathcolor#1#2#3{%
	\protect\leavevmode
	\begingroup
		\color#1{#2}#3%
	\endgroup
}
\makeatother


% NOTE: Linguaggi formali e compilatori
\newcommand\produce{\longrightarrow}
\newcommand\deriva{\Rightarrow}

\usepackage{stackrel}

% TODO: simbolo spostato sulla destra
\newcommand\derivamultiplo{\stackrel{\ensuremath{+}}{\deriva}}
\newcommand\derivanumero[1]{\stackrel{\ensuremath{#1}}{\deriva}}

% NOTE: analisi
\renewcommand\restriction{\mathord{\upharpoonright}}

% NOTE: "uq" sta per "upquote"
\newcommand\uq{\ensuremath \text{\textquotesingle}}

% NOTE: abbreviazioni
\newcommand\ob{\overbrace}
\newcommand\ub{\underbrace}
\newcommand\us{\underset}
\newcommand\ul{\underline}
\newcommand\tn{\tn}
\newcommand\fns{\footnotesize}

% NOTE: simbolo di fine dimostrazione
\renewcommand\qedsymbol{\( \blacksquare \)}

% NOTE: teoremi, lemmi e nota bene
\newtheorem{theorem}{Teorema}
\newtheorem{lemma}[theorem]{Lemma}
\newtheorem{definition}{Definizione}
\newtheorem{corollario}{Corollario}
\newtheorem{example}{Esempio}
\newtheorem{proposition}{Proposizione}

\newtheorem*{remark}{Ricorda}
\newtheorem*{note}{Nota}
\newtheorem*{observation}{Osservazione}
\newtheorem*{hint}{Suggerimento}
