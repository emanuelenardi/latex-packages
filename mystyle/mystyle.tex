% ---------------------------------------------------------------------------- %
% ---------------------------------------------------------------------------- %
% ---------------------------------------------------------------------------- %

% NOTE: Abbreviazioni LFC
\newcommand{\vuoto}{\emptyset}
% TODO: Dispensa di LFC
% \newcommand{\e}{\epsilon}

\newcommand{\produce}{\longrightarrow}
\newcommand{\deriva}{\implies}
\newcommand{\derivamultiplo}{\underset{\deriva}{+}}		% Derivazione multipla
\newcommand{\derivanumero}[1]{\underset{\deriva}{#1}}	% Segna il no. di passaggi

% TODO: Dispensa di LFC
% \newcommand{\uq}{\upquote}
\newcommand{\upquote}{\text{\textquotesingle}}

\newcommand{\ob}{\overbrace}
\newcommand{\ub}{\underbrace}

% ---------------------------------------------------------------------------- %
% ---------------------------------------------------------------------------- %

% NOTE: Abbreviazioni simboli matematici
\newcommand{\N}{\mathbb{N}}	% natural
\newcommand{\Z}{\mathbb{Z}}	% integers
\newcommand{\I}{\mathbb{I}}	% irrational
\newcommand{\Q}{\mathbb{Q}}	% rational
\newcommand{\R}{\mathbb{R}}	% real
\newcommand{\C}{\mathbb{C}}	% complex

\renewcommand\qedsymbol{\( \blacksquare \)}

\newtheorem{theorem}{Teorema}
\newtheorem{lemma}[theorem]{Lemma}
\newtheorem{definition}{Definizione}

\newtheorem*{remark}{Ricorda}
\newtheorem*{note}{Nota}

\DeclarePairedDelimiter{\norm}{\lVert}{\rVert}
\DeclarePairedDelimiter{\abs}{\lvert}{\rvert}
\DeclarePairedDelimiter\Bracket{\lbrack}{\rbrack}

\newcolumntype{C}{>{$}c<{$}}
\newcolumntype{L}{>{$}l<{$}}
\newcolumntype{R}{>{$}r<{$}}

\makeatletter
\def\mathcolor#1#{\@mathcolor{#1}}
\def\@mathcolor#1#2#3{%
	\protect\leavevmode
	\begingroup
		\color#1{#2}#3%
	\endgroup
}
\makeatother

% ---------------------------------------------------------------------------- %
% ---------------------------------------------------------------------------- %

% NOTE: Permette di inserisce una pagina bianca
\newcommand{\blankpage}{%
	\null%
	\thispagestyle{empty}%
	\addtocounter{page}{-1}%
	\newpage%
}

% NOTE: Definizione della data ->
\newdateformat{mydate}{%
	\THEDAY{ }\monthname[\THEMONTH] \THEYEAR%		% Definisce il formato della data
}

\date{\today}
\newcommand{\dateandtime}{{\mydate\today}}

% NOTE: Simbolo per riferimento a materiali esterni
\newcommand{\ExternalLink}{
	\tikz[x = 1.2ex, y = 1.2ex, baseline = -0.05ex]{
		\begin{scope}[x = 1ex, y = 1ex]
			\clip (-0.1,-0.1) --++ (-0, 1.2) --++ (0.6, 0) --++ (0, -0.6) --++ (0.6, 0) --++ (0, -1);
			\path[draw, line width = 0.5, rounded corners = 0.5] (0,0) rectangle (1,1);
		\end{scope}
		\path[draw, line width = 0.5] (0.5, 0.5) -- (1, 1);
		\path[draw, line width = 0.5] (0.6, 1) -- (1, 1) -- (1, 0.6);
	}
}

\newcommand{\lstinputpath}[1]{\lstset{inputpath=#1}}

\newcommand{\mail}[1]{\href{mailto:#1}{\texttt{#1}}}

\newcommand{\mysection}[2]{\section[#1]{#1\\[.5ex]\normalsize\textit{#2}}}
\newcommand{\mysubsection}[2]{\subsection[#1]{#1\\[.5ex]\normalsize\textit{#2}}}

\newcommand{\omissis}{[\textellipsis\unkern]}
\newcommand{\code}[1]{\texttt{\textbf{#1}}}

% ---------------------------------------------------------------------------- %
% ---------------------------------------------------------------------------- %

% NOTE: Abbreviazioni inserimento codice
\newcommand{\html}[1]{\lstinline[style = HTML]|#1|}
\newcommand{\cc}[1]{\lstinline[style = C]|#1|}
\newcommand{\cpp}[1]{\lstinline[style = [11]C++]|#1|}
\newcommand{\gradle}[1]{\lstinline[style = Gradle]|#1|}
\newcommand{\java}[1]{\lstinline[style = Java]|#1|}
\newcommand{\javascript}[1]{\lstinline[style = Javascript]|#1|}
\newcommand{\jsp}[1]{\lstinline[style = JSP]|#1|}
\newcommand{\sml}[1]{\lstinline[style = SML]|#1|}
\newcommand{\sql}[1]{\lstinline[style = SQL]|#1|}
\newcommand{\xml}[1]{\lstinline[style = XML]|#1|}

% ---------------------------------------------------------------------------- %
% ---------------------------------------------------------------------------- %

% NOTE: Linea spessa sul testo sottostante
\newcommand{\soutthick}[1]{
	\renewcommand{\ULthickness}{1.0pt}	% 2.4
		\sout{#1}%
	\renewcommand{\ULthickness}{.4pt}	% Resetting to ulem default
}

% ---------------------------------------------------------------------------- %
% ---------------------------------------------------------------------------- %

% NOTE: Definizione nuovi colori
\definecolor{ashgrey}{rgb}{0.7, 0.75, 0.71}
\definecolor{burgundy}{rgb}{0.5, 0.0, 0.13}
\definecolor{cyan}{rgb}{0.0,0.6,0.6}
\definecolor{darkblue}{rgb}{0.0,0.0,0.6}
\definecolor{gray}{rgb}{0.4,0.4,0.4}

% NOTE: Definizione nomi colori personalizzati
\colorlet{selection}{red}
\colorlet{purple}{MediumPurple1}
\colorlet{greenYellow}{Chartreuse2}

% NOTE: Abbreviazioni colori
\newcommand{\darkblue}[1]{\textcolor{darkblue}{#1}}
\newcommand{\cyan}[1]{\textcolor{cyan}{#1}}
\newcommand{\gray}[1]{\textcolor{gray}{#1}}

\newcommand{\blue}[1]{\textcolor{blue}{#1}}
\newcommand{\azure}[1]{\textcolor{SkyBlue}{#1}}
\newcommand{\green}[1]{\textcolor{ForestGreen}{#1}}
\newcommand{\greenlight}[1]{\textcolor{green!30}{#1}}
\newcommand{\red}[1]{\textcolor{red}{#1}}
\newcommand{\orange}[1]{\textcolor{red!50}{#1}}
\newcommand{\purple}[1]{\textcolor{purple}{#1}}

\newcommand{\greenHL}{\btHL[fill=green!30]}				% Evidenzia in verde
\newcommand{\blueHL}{\btHL[fill=blue!30]}				% Evidenzia in blue
\newcommand{\azureHL}{\btHL[fill=SkyBlue]}				% Evidenzia in blue
\newcommand{\redHL}{\btHL}								% Evidenzia in rosso
\newcommand{\yellowHL}{\btHL[fill=yellow!60]}			% Evidenzia in giallo
\newcommand{\orangeHL}{\btHL[fill=orange!60]}

% NOTE: Marking
\newcommand{\cmark}{\green{\ding{51}}}					% \verb|\cmark|: \cmark \par
\newcommand{\xmark}{\red{\ding{55}}}					% \verb|\xmark|: \xmark

% ---------------------------------------------------------------------------- %
% ---------------------------------------------------------------------------- %

% TODO: Rimuovila da tutte le dispense, da ri-definire
% \renewcommand{\descriptionlabel}[1]{\hspace{\labelsep}\texttt{#1}}

% NOTE: Imposta il trattino negli elenchi puntati come marcatore
\renewcommand{\labelitemi}{--}
% NOTE: Diminuisce la distanza fra i punti
\setlist[1]{itemsep = 3pt, topsep = 3pt}

% NOTE: Rinomina la didascalia dei codici
\renewcommand{\lstlistingname}{Codice}					% Listing -> Codice
% NOTE: Rinomina l'indice dei codici
\renewcommand{\lstlistlistingname}{Lista dei Codici}	% List of Listings -> Lista dei Codici

% ---------------------------------------------------------------------------- %
% ---------------------------------------------------------------------------- %

% HACK: Change it when you use "book" as document class
% \renewcommand{\cftchapafterpnum}{\vspace{10pt}}
\renewcommand{\cftsecafterpnum}{\vspace{10pt}}

% \let\Chapter\chapter
% \def\chapter{\addtocontents{lol}{\protect\addvspace{10pt}}\Chapter}
\let\Section\section%
\def\section{\addtocontents{lol}{\protect\addvspace{10pt}}\Section}

% NOTE: La numerazione delle sezione si azzera all'inizio di una nuova parte
\counterwithin*{section}{part}

\setlist[description]{leftmargin = 8em, style = nextline}
% NOTE: toglie l'intentazione a tutto il documento
\setlength\parindent{0pt}

% ---------------------------------------------------------------------------- %
% ---------------------------------------------------------------------------- %

% NOTE: Definizione nuovi comandi per il pacchetto "notes"
\newcommandx{\insicuro}[2][1=]{\todo[linecolor=red,backgroundcolor=red!25,bordercolor=red,#1]{#2}}
\newcommandx{\cambia}[2][1=]{\todo[linecolor=blue,backgroundcolor=blue!25,bordercolor=blue,#1]{#2}}
\newcommandx{\info}[2][1=]{\todo[linecolor=OliveGreen,backgroundcolor=OliveGreen!25,bordercolor=OliveGreen,#1]{#2}}
\newcommandx{\miglioramento}[2][1=]{\todo[linecolor=Plum,backgroundcolor=Plum!25,bordercolor=Plum,#1]{#2}}
\newcommandx{\nascosto}[2][1=]{\todo[disable,#1]{#2}}

% ---------------------------------------------------------------------------- %
% ---------------------------------------------------------------------------- %

% NOTE: Definizione nuovo ambiente per evidenziare il codice
\makeatletter
	\newenvironment{btHighlight}[1][]
		{\begingroup\tikzset{bt@Highlight@par/.style={#1}}\begin{lrbox}{\@tempboxa}}
		{\end{lrbox}\bt@HL@box[bt@Highlight@par]{\@tempboxa}\endgroup}

	\newcommand\btHL[1][]{%
		\begin{btHighlight}[#1]\bgroup\aftergroup\bt@HL@endenv%
	}
	\def\bt@HL@endenv{%
		\end{btHighlight}%
		\egroup%
	}
	\newcommand{\bt@HL@box}[2][]{%
		\tikz[#1]{%
			\pgfpathrectangle{\pgfpoint{1pt}{0pt}}{\pgfpoint{\wd #2}{\ht #2}}%
			\pgfusepath{use as bounding box}%
			\node[anchor=base west, fill=orange!30,outer sep=0pt,inner xsep=1pt, inner ysep=0pt, rounded corners=3pt, minimum height=\ht\strutbox+1pt,#1]{\raisebox{1pt}{\strut}\strut\usebox{#2}};
		}%
	}
\makeatother

% ---------------------------------------------------------------------------- %
% ---------------------------------------------------------------------------- %

% NOTE: Impostazione delle didascalie
\captionsetup{
	figureposition = bottom,	% opzione analoga alla successiva per le figure
	tableposition = top,		% ordina al programma di inserire uno spazio adeguato tra didascalia e tabella
	% position = bottom,			% Non funziona
	font = small,				% produce didascalie in corpo più piccolo
	format = hang,				% allinea (hang) alla prima riga quelle successive
	labelfont = {sf,bf}			% imposta l’etichetta della didascalia in caratteri senza grazie, in grassetto
}

% NOTE: Ridefinisce le virgolette
\DeclareQuoteStyle{italian}%
	{\textquotedblleft}
	[\textquotedblleft]
	{\textquotedblright}
		[0.05em]
	{\textquoteleft}
	[\textquoteleft]
	{\textquoteright}

% NOTE: Impsta il tipo di virgolette
\setquotestyle{italian}

% ---------------------------------------------------------------------------- %
% ---------------------------------------------------------------------------- %

% WARNING: può creare probemi a tempo di compilazione
\hypersetup{
	pdfinfo = {
		pdffitwindow = true,	 		% window fit to page when opened
		pdfnewwindow = true,			% links in new PDF window
		pdftitle = {\pdfTitle},			% PDF's title
		pdfsubject = {\subject},		% subject of the document
		pdfkeywords = {\tags},			% list of keywords
		pdfauthor = {\authorName},		% author of the document
		pdfcreator = {\authorName},		% creator of the document
		pdfproducer = {\authorName},	% producer of the document
	}
}

\geometry{
	paper = a4paper,					% formato carta A4
	% margin = 2cm,               		% tutti i margini
	top = 1.5cm,						% margine superiore
	bottom = 2cm,						% margine inferiore
	left = 2cm,							% margine sinistro
	right = 2cm,						% margine destro
}

% ---------------------------------------------------------------------------- %
% ---------------------------------------------------------------------------- %

% HACK: Commenta quando usi la documentclass "exam"
% \begin{comment}

\fancypagestyle{footer}{
	\fancyhf{}
	\lfoot{\authorName}
	\cfoot{\footnotesize Pagina~\thepage\ di~\pageref{LastPage}}
	\rfoot{\dateandtime}
	\renewcommand{\headrulewidth}{0pt}
	\renewcommand{\footrulewidth}{2pt} % 0.8pt
}

% \begin{comment}

% ---------------------------------------------------------------------------- %
% ---------------------------------------------------------------------------- %

\newcolumntype{!}{>{\global\let\currentrowstyle\relax}}
\newcolumntype{^}{>{\currentrowstyle}}
\newcommand{\rowstyle}[1]{\gdef\currentrowstyle{#1}%
	#1\ignorespaces%
}

% ---------------------------------------------------------------------------- %
% ---------------------------------------------------------------------------- %

% FIXME: Fare attenzione, potrebbe creare problemi
\newcommand{\nodeDistance}{2.0cm}
\newcommand{\nextto}{0.5cm}

% ---------------------------------------------------------------------------- %

% NOTE: Digramma Entità-Relazione
\tikzset{%
	every entity/.style = {draw = orange, fill = orange!20},
	every attribute/.style = {draw = purple, fill = purple!20, node distance = 1cm},
	every relationship/.style = {draw = greenYellow, fill = greenYellow!20},
	every isa/.style = {draw = green, fill = green!20},
	every database/.style = {draw = blue, fill = blue!20},
}

\tikzset{%
	between/.style args={#1 and #2}{%
		at = (\( (#1)!0.5!(#2) \))%
	},
}

% ---------------------------------------------------------------------------- %

% NOTE: Flow-diagram
\tikzstyle{startstop} = [rectangle, rounded corners, minimum width = 3cm, minimum height = 1cm, text centered, every startstop]
\tikzstyle{io} = [trapezium, trapezium left angle = 70, trapezium right angle = 110, minimum width = 3cm, minimum height = 1cm, text centered, every io]
\tikzstyle{process} = [rectangle, minimum width = 3cm, minimum height = 1cm, text centered, text width = 3cm, every process]
\tikzstyle{decision} = [diamond, minimum width = 3cm, minimum height = 1cm, text centered, text width = 3cm, every decision]
\tikzstyle{arrow}  = [every arrow]

\tikzstyle{every startstop}  = [draw = black, fill = red!30]
\tikzstyle{every io}  = [draw = black, fill = blue!30]
\tikzstyle{every process}  = [draw = black, fill = orange!30]
\tikzstyle{every decision}  = [draw = black, fill = green!30]
\tikzstyle{every arrow}  = [thick, ->, >=stealth]

% NOTE: Images
\tikzstyle{rettangolo} = [rectangle, draw, thick, minimum width = 4cm, minimum height = 2cm, text centered]
\tikzstyle{freccia} = [single arrow, draw, thick, minimum width = 1cm, minimum height = 4cm, every arrow]
\tikzstyle{freccia doppia} = [double arrow, draw, thick, minimum width = 1cm, minimum height = 4cm, every arrow]

% ---------------------------------------------------------------------------- %
% ---------------------------------------------------------------------------- %

% NOTE: Dichiarazione dei percorsi DI DEFAULT per le immagini
\graphicspath{
	{./assets/figures/}
	{./assets/slides/}
	{./assets/icons/}
}

% NOTE: Dichiarazione dei percorsi DI DEFAULT per il codice
\lstinputpath{./assets/codes/}

% ---------------------------------------------------------------------------- %
% ---------------------------------------------------------------------------- %
% ---------------------------------------------------------------------------- %
