% ---------------------------------------------------------------------------- %
% ---------------------------------------------------------------------------- %
% ---------------------------------------------------------------------------- %
% Welcome to hell!
%
% If you are editing this, look. I'm sorry.
% I did not expect you to come here.
% What are doing in this barren wasteland anyway?
% Well, I guess if you've made it this far, you've grown to understand my ugly code.
% But since this is (probably) the worst part, feel free to ask me for help:
%
% emanuele.nardi@studenti.unitn.it
%
% Note to future self:
% This message does not apply to you. Enjoy hell.
% ---------------------------------------------------------------------------- %
% ---------------------------------------------------------------------------- %
% ---------------------------------------------------------------------------- %

% NOTE: "a4paper" defines the paper size;
% NOTE: "11pt" dets the size of the main font in the document. If no option is specified, 10pt is assumed.
% NOTE: specifies whether a new page should be started after the document title or not. The article class does not start a new page by default, while report and book do.
% NOTE: "draft" makes LaTeX indicate hyphenation and justification problems with a small square in the right-hand margin of the problem line so they can be located quickly by a human. It also suppresses the inclusion of images and shows only a frame where they would normally occur;
\documentclass[
	,a4paper
	,10pt
	,titlepage
	% ,draft
]{article}

% ---------------------------------------------------------------------------- %
% ---------------------------------------------------------------------------- %
\usepackage{comment}

% NOTE: The cmap package is intended to make the PDF files generated by pdflatex
% "searchable and copyable" in acrobat reader and other compliant PDF viewers.
% \usepackage{cmap}
% NOTE: % cmap + mathematics (Unicode)
\usepackage[noTeX]{mmap}

\usepackage{ifxetex}
\ifxetex
	% NOTE: compilation with XeLaTeX
	\usepackage[no-math]{fontspec}

	% NOTE: babel for XeTeX
	\usepackage{polyglossia}
	\setmainlanguage{italian}

	\usepackage{xltxtra}
	\newcommand{\tex}{\TeX}
	\newcommand{\latex}{\LaTeX}
	\newcommand{\xelatex}{\XeLaTeX}

	% TODO: specifica font predefiniti con xelatex
\else
	% NOTE: compilation with LaTeX
	% NOTE: font encription
	\usepackage[T1]{fontenc}
	\usepackage[utf8]{inputenc}
	% NOTE: languages used in the document
	% NB: l'ultima dev'essere la lingua principale del documento
	\usepackage[english, main=italian]{babel}

	% NOTE: introdotto dal progetto calliope
	% \usepackage[default, osfigures, scale=0.95]{opensans}
	\usepackage[scaled=.8]{sourcecodepro}

	% IDEA: "microtype" - migliora il riempimento delle righe
	% NOTE: activate = {true,nocompatibility} - activate protrusion and expansion
	% NOTE: final - enable microtype; use "draft" to disable
	% NOTE: tracking = true, kerning=true, spacing=true - activate these techniques
	% NOTE: factor = 1100 - add 10% to the protrusion amount (default is 1000)
	% NOTE: stretch = 10, shrink = 10 - reduce stretchability/shrinkability (default is 20/20)
	\usepackage[
		% ,activate	= {true,nocompatibility}
		,activate	= {true}
		,final
		% ,draft
		,tracking	= true
		,kerning	= true
		,spacing	= true
		,factor		= 1100
		,stretch	= 10
		,shrink		= 10
	]{microtype}
\fi

% ---------------------------------------------------------------------------- %
% ---------------------------------------------------------------------------- %

% NOTE: "xcolor" - permette di definire colori personalizzati
\usepackage[
	,dvipsnames
	,svgnames
	,x11names
	,table
	% ,gray
]{xcolor}


\colorlet{bodyColor}{blue!50!white}
\colorlet{driverColor}{blue}

% \colorlet{alternative}{red!50!white}

\colorlet{rose}{red!50!white}
% \colorlet{accent}{red!70}

% NOTE: dichiarato in set-math
\colorlet{brighter}{blue!50!white}
\colorlet{darker}{blue!50!black}

\newcommand\darker[1]{{\color{darker} #1}}

% NOTE: Definizione nuovi colori
% \definecolor{burgundy}{rgb}{0.5, 0.0, 0.13}
% \definecolor{cyan}{rgb}{0.0,0.6,0.6}
% \definecolor{darkblue}{rgb}{0.0,0.0,0.6}
% \definecolor{gray}{rgb}{0.4,0.4,0.4}

% NOTE: TODO
% \colorlet{accentColor}{blue}
% \colorlet{purple}{MediumPurple1}
% \colorlet{greenYellow}{Chartreuse2}

% NOTE: Abbreviazioni colori
% \newcommand{\darkblue}[1]{\textcolor{darkblue}{#1}}
% \newcommand{\cyan}[1]{\textcolor{cyan}{#1}}
% \newcommand{\gray}[1]{\textcolor{gray}{#1}}

% NOTE: "pifont" - introduce i simboli \cmark e \xmark
\usepackage{pifont}

% NOTE: Marking
\newcommand{\cmark}{\textcolor{ForestGreen}{\ding{51}}}
\newcommand{\xmark}{\textcolor{Red}{\ding{55}}}

\newcommand{\tick}{\cmark}

% OPTIMIZE: rimuovere dai documenti
% \newcommand{\blue}[1]{\textcolor{blue}{#1}}
% \newcommand{\azure}[1]{\textcolor{SkyBlue}{#1}}
% \newcommand{\green}[1]{\textcolor{ForestGreen}{#1}}
% \newcommand{\greenlight}[1]{\textcolor{green!30}{#1}}
% \newcommand{\red}[1]{\textcolor{red}{#1}}
% \newcommand{\orange}[1]{\textcolor{red!50}{#1}}
% \newcommand{\purple}[1]{\textcolor{purple}{#1}}

% NOTE 'datetime2' - gestione e stampa delle date in vari formati
\usepackage[useregional,showdow]{datetime2}

% NOTE: gestisce la dimensione massima delle immagini
\usepackage[export]{adjustbox}

% IDEA: figure
% NOTE: "graphicx" - inserimento figure nel docuemento
% NOTE: "float" - definisce l'opzione "H" per gli oggetti fluttuanti
% NOTE: "wrapfig" - inserimento di figure di fianco al testo
\usepackage{
	graphicx,
	float,
	subcaption,
	wrapfig,
}

% NOTE: Dichiarazione dei percorsi DI DEFAULT per le immagini
\graphicspath{
	{./assets/figures/}
	{./assets/slides/}
	{./assets/icons/}
}


% WARNING: package deprecatig
% OPTIMIZE: footmisc -> TODO
% OPTIMIZE: subfig -> subcaption
% OPTIMIZE: caption -> subcaption

% NOTE: "algorithm2e" - specifica degli algoritmi
% NOTE: "algpseudocode" - speudocodice
% NOTE: "alltt" - ridefinisce l'ambiente "verbatim"
\usepackage{
	algorithmic,
	algpseudocode,
}
% NOTE: from https://www.thomasdenney.co.uk/blog/2017/4/18/typesetting-algorithms-with-latex/
\algtext*{EndWhile}
\algtext*{EndFor}
\algtext*{EndIf}
\algtext*{EndFunction}

\algnewcommand{\SIf}[1]{\State\algorithmicif\ #1\ \algorithmicthen}
\algnewcommand{\SElseIf}[1]{\State\algorithmicelse\ \algorithmicif\ #1\ \algorithmicthen}
\algnewcommand{\SElse}{\State\algorithmicelse\ }
\algnewcommand{\SWhile}[1]{\State\algorithmicwhile\ #1\ \algorithmicdo}
\algnewcommand{\SFor}[1]{\State\algorithmicfor\ #1\ \algorithmicdo}
\algnewcommand{\SForAll}[1]{\State\algorithmicforall\ #1\ \algorithmicdo}

\captionsetup{
	figureposition = bottom,
	tableposition = bottom,
	font = small,
	labelfont = {sf,bf},
	justification = centering,
}

\newcolumntype{C}{>{$}c<{$}}
\newcolumntype{L}{>{$}l<{$}}
\newcolumntype{R}{>{$}r<{$}}

% OPTIMIZE: ci sono modi migliori per farlo
\newcolumntype{!}{>{\global\let\currentrowstyle\relax}}
\newcolumntype{^}{>{\currentrowstyle}}
\newcommand{\rowstyle}[1]{\gdef\currentrowstyle{#1}%
	#1\ignorespaces%
}
%
% NOTE 'geometry' - gestisce i margini della pagina
\usepackage{geometry}

% NOTE 'paper' - formato carta A4
% NOTE 'margin' - specifica tutti i margini
% NOTE 'heightrounded' - avoiding cases of "underful vbox"
% https://tex.stackexchange.com/questions/123291/
\geometry{
	% ,margin	= 2cm
	,top	= 1.5cm
	,bottom	= 2cm
	,left	= 2cm
	,right	= 2cm
	,heightrounded
}

% NOTE 'ulem' - permette di avere diversi tipi di sottolineatura
% NB 'normalem' - replaces underlining with italics in text emphasized by \emph
\usepackage[normalem]{ulem}

% NOTE Linea spessa sul testo sottostante
\newcommand{\soutthick}[1]{
	\renewcommand{\ULthickness}{1.0pt}	% 2.4
		\sout{#1}%
	\renewcommand{\ULthickness}{.4pt}	% Resetting to ulem default
}

% NOTE 'mdframed' - TODO
\usepackage[framemethod = tikz]{mdframed}

% NOTE Definizione nuovo ambiente per evidenziare il codice
% \makeatletter
% 	\newenvironment{btHighlight}[1][]
% 		{\begingroup\tikzset{bt@Highlight@par/.style={#1}}\begin{lrbox}{\@tempboxa}}
% 		{\end{lrbox}\bt@HL@box[bt@Highlight@par]{\@tempboxa}\endgroup}
%
% 	\newcommand\btHL[1][]{%
% 		\begin{btHighlight}[#1]\bgroup\aftergroup\bt@HL@endenv%
% 	}
% 	\def\bt@HL@endenv{%
% 		\end{btHighlight}%
% 		\egroup%
% 	}
% 	\newcommand{\bt@HL@box}[2][]{%
% 		\tikz[#1]{%
% 			\pgfpathrectangle{\pgfpoint{1pt}{0pt}}{\pgfpoint{\wd #2}{\ht #2}}%
% 			\pgfusepath{use as bounding box}%
% 			\node[anchor=base west, fill=orange!30,outer sep=0pt,inner xsep=1pt, inner ysep=0pt, rounded corners=3pt, minimum height=\ht\strutbox+1pt,#1]{\raisebox{1pt}{\strut}\strut\usebox{#2}};
% 		}%
% 	}
% \makeatother

% \newcommand{\greenHL}{\btHL[fill=green!30]}
% \newcommand{\blueHL}{\btHL[fill=blue!30]}
% \newcommand{\azureHL}{\btHL[fill=SkyBlue]}
% \newcommand{\redHL}{\btHL}
% \newcommand{\yellowHL}{\btHL[fill=yellow!60]}
% \newcommand{\orangeHL}{\btHL[fill=orange!60]}

% \input{set-listing}
% NOTE 'enumitem' - permette di personalizzare gli elenchi puntati
% NOTE lista in linea
\usepackage[inline]{enumitem}

% NOTE imposta il trattino negli elenchi puntati come marcatore
% \renewcommand{\labelitemi}{\normalfont\bfseries\textendash}

% NOTE diminuisce GLOBALMENTE la distanza fra i punti
\setlist{
	% itemsep = 3pt, % default: 3pt
	% topsep = 3pt   % default: >3pt
}

% NOTE definizione liste personalizzate
% \newlist{<name>}{<type>}{<max-depth>}

\newlist{compactlist}{itemize}{2}
\setlist[compactlist]{label=\textbullet, noitemsep, topsep = 0pt, parsep = 0pt, partopsep = 0pt}

\newlist{semicompactlist}{itemize}{2}
\setlist[semicompactlist]{label=\textbullet, noitemsep, topsep = 0pt, parsep = 5pt, partopsep = 0pt}

% NB dipendenza 'set-tikz-macros'
\newlist{circledlist}{enumerate}{10}
\setlist[circledlist]{label=\protect\circled{\arabic*}}

% IDEA annotazioni - note a margine
% TODO usalo per i commenti nella tesi
\usepackage{
	% todonotes,
	marginnote,
	mparhack,
	marginfix,
}

% WARNING: Mantenee l'ordine dei pacchetti è fonadamentale per non rompere la build
% WARNING: "semantic" da caricare sopo "amsmath"
% IDEA: math suymbols
% NOTE: "amsthm" - teoremi e dimostrazioni
% NOTE: "amsfonts" - nomi insiemi numerici
% NOTE: "amssymb" - leqslant & geqslant
% NOTE: "mathtools" - mathtools = amsmath + other stuff
% NOTE: "MnSymbol" - fornisce le freccie che utilizzo per andare a capo riga nelle liste di codice
% NOTE: "abraces" - angle brakets fine tuning
% NOTE: "braket" - permette l'uso di parentesi angolari
% NOTE: "nicefrac" - divisione in linea
% NOTE: "textgreek" - caratteri greci
% NOTE: "siunitx" - unità di misura del SI
\usepackage{
	,amsthm
	,amsfonts
	,amssymb
	,MnSymbol
	,mathtools
	% ,braket
	,semantic
	,empheq
	,nicefrac
}


% NOTE: comando definito dal package "amsmath"
\DeclarePairedDelimiter\norm{\lVert}{\rVert}
\DeclarePairedDelimiter\abs{\lvert}{\rvert}
\DeclarePairedDelimiter\Abs{\bigg\lvert}{\bigg\rvert}
\DeclarePairedDelimiter\Bracket{\lbrack}{\rbrack}

% NOTE: colorare simboli matematici
% tex.stackexchange.com/questions/21598/
\makeatletter
\def\mathcolor#1#{\@mathcolor{#1}}
\def\@mathcolor#1#2#3{%
	\protect\leavevmode
	\begingroup
		\color#1{#2}#3%
	\endgroup
}
\makeatother


% NOTE: Linguaggi formali e compilatori
\newcommand\produce{\longrightarrow}
\newcommand\deriva{\Rightarrow}

\usepackage{stackrel}

% TODO: simbolo spostato sulla destra
\newcommand\derivamultiplo{\stackrel{\ensuremath{+}}{\deriva}}
\newcommand\derivanumero[1]{\stackrel{\ensuremath{#1}}{\deriva}}

% NOTE: analisi
\renewcommand\restriction{\mathord{\upharpoonright}}

% NOTE: "uq" sta per "upquote"
\newcommand\uq{\ensuremath \text{\textquotesingle}}

% NOTE: abbreviazioni
\newcommand\ob{\overbrace}
\newcommand\ub{\underbrace}
\newcommand\us{\underset}
\newcommand\ul{\underline}
\newcommand\tn{\tn}
\newcommand\fns{\footnotesize}

% NOTE: simbolo di fine dimostrazione
\renewcommand\qedsymbol{\( \blacksquare \)}

% NOTE: teoremi, lemmi e nota bene
\newtheorem{theorem}{Teorema}
\newtheorem{lemma}[theorem]{Lemma}
\newtheorem{definition}{Definizione}
\newtheorem{corollario}{Corollario}
\newtheorem{example}{Esempio}
\newtheorem{proposition}{Proposizione}

\newtheorem*{remark}{Ricorda}
\newtheorem*{note}{Nota}
\newtheorem*{observation}{Osservazione}
\newtheorem*{hint}{Suggerimento}

% IDEA gestione delle pagine

% NOTE 'pdfpages' inserimento di pdf all'interno del documento
% NOTE 'afterpage' esegue il comando dopo la prossima interruzione di pagine
\usepackage{
	,pdfpages
	,afterpage
}

% NOTE permette di inserire una pagina bianca
\newcommand{\blankpage}{%
	\null%
	\thispagestyle{empty}%
	\addtocounter{page}{-1}%
	\newpage%
}

% NOTE toglie i numeri a piè pagina dalle pagine bianche
\usepackage{emptypage}

% NOTE 'fancyhdr' permette la personalizzazione di testatina e piè di pagina
% NOTE 'lastpage' riferimento all'ultima pagina
\usepackage{fancyhdr, lastpage}

% NOTE toglie l'intentazione a tutto il documento
\setlength{\parindent}{0ex}
\setlength{\parskip}{1ex}

% NOTE: Ridefinisce le virgolette
\DeclareQuoteStyle{italian}%
	{\textquotedblleft}
	[\textquotedblleft]
	{\textquotedblright}
		[0.05em]
	{\textquoteleft}
	[\textquoteleft]
	{\textquoteright}

% NOTE: Impsta il tipo di virgolette
\setquotestyle{italian}

% IDEA: rotazione
% NOTE: "pdfpages" - inserimento di pdf all'interno del documento
% NOTE: "rotating" - rotazione tabelle
\usepackage{
	pdflscape,
	rotating,
}

% IDEA: tabelle
% NOTE: "array" - permette di creare delle colonne personalizzate
% NOTE: "bigstrut" -
% NOTE: "booktabs" - genera filetti professionali per le tabelle
% NOTE: "colortbl" - righe e celle colorate
% NOTE: "diagbox" - diagonal rule on a cell
% NOTE: "ltablex" - crea tabelle dalla larghezza dinamica su più pagine
% NOTE: "makecell" -
% NOTE: "multirow" - tabelle con righe multilinea
% NOTE: "tabularx" - crea tabelle dalla larghezza dinamica
\usepackage{
	array,
	tabu,
	xcolor,
	bigstrut,
	booktabs,
	colortbl,
	diagbox,
	ltablex,
	makecell,
	multirow,
	tabularx,
}
\setlength\lightrulewidth{0.1pt}

% NOTE: optional
% tex.stackexchange.com/questions/177164/
% tex.stackexchange.com/questions/341656/
\usetikzlibrary[external]
% \usepackage{pgfplots}
% \usepgfplotslibrary[
% 	external
% ]

% Macro holding the externalized sub-directory
\newcommand{\externaldirectory}{_tikz-cache/}

% NOTE: opzioni per la pre-compilazioni delle immagini create con tikz
\tikzexternalize[
	mode = graphics if exists,
	% figure list = true,
	% All externalized graphics go to the \externaldirectory
	prefix = \externaldirectory
]
% Externalise only on-demand.
\tikzexternaldisable

% IDEA Disegare grafici
% NOTE 'tikz' - pacchetto completo per disegnare su LaTeX
\usepackage{tikz}

% NOTE sottolibrerie del pacchetto tikz
\usetikzlibrary{
	,arrows
	,calc
	,intersections
	,matrix
	,positioning
	,shapes.geometric
	,tikzmark
	,trees
	,decorations.text
	,decorations.pathmorphing
	,decorations.pathreplacing
}

\usepackage{set-tikz-macros}

\newcommand{\newState}[4]{\node[state,#3](#1)[#4]{#2};}
\newcommand{\newTransition}[4]{\path[->] (#1) edge [#4] node {#3} (#2);}

\tikzset{
	node distance = 1cm and 1cm,
	initial distance = 0.5cm,
	initial text = {\emph{inizio}},
	double distance = 1.5pt,
	every state/.style = {
		draw,
		fill = gray!10,
		minimum size = 10mm
	},
	every edge/.style = {
		draw,
		->, >=stealth',
		shorten >= 1pt,
		thin,
		on grid,
	}
}

% NOTE: "algorithm2e" - specifica degli algoritmi
% NOTE: "algpseudocode" - speudocodice
% NOTE: "alltt" - ridefinisce l'ambiente "verbatim"
\usepackage{
	algorithmic,
	algpseudocode,
}
% NOTE: from https://www.thomasdenney.co.uk/blog/2017/4/18/typesetting-algorithms-with-latex/
\algtext*{EndWhile}
\algtext*{EndFor}
\algtext*{EndIf}
\algtext*{EndFunction}

\algnewcommand{\SIf}[1]{\State\algorithmicif\ #1\ \algorithmicthen}
\algnewcommand{\SElseIf}[1]{\State\algorithmicelse\ \algorithmicif\ #1\ \algorithmicthen}
\algnewcommand{\SElse}{\State\algorithmicelse\ }
\algnewcommand{\SWhile}[1]{\State\algorithmicwhile\ #1\ \algorithmicdo}
\algnewcommand{\SFor}[1]{\State\algorithmicfor\ #1\ \algorithmicdo}
\algnewcommand{\SForAll}[1]{\State\algorithmicforall\ #1\ \algorithmicdo}

% TODO: https://it.overleaf.com/learn/latex/Writing_your_own_package
% \RequirePackage{algorithm2e}

% NOTE Le seguenti domande hanno influenzato le impostazioni
% https://tex.stackexchange.com/questions/345737/

\usepackage{comment} % debugging
% NOTE impostazioni della stampa degli algoritmi
\usepackage[
	% ,linesnumbered
	% ,plain% default
	,ruled
	% ,tworuled
	% ,noline% default
	,vlined
	% ,italiano
	% ,onelanguage
	,algosection % default
	% ,algo2e % per compatibilità con 'algorithm' TODO ambiente algorithm -> algorithm2e
]{algorithm2e}
\DontPrintSemicolon
\LinesNotNumbered

% NOTE inclusione delle immagini esplicative negli algoritmi
\newif\ifFigureOfAlgo
\FigureOfAlgotrue% stampa l'immagine
% NOTE ottimizzare la versione stampata in bianco e nero
\newif\ifAlgoPrinted
% TODO rimuovere condizionalmente i commenti dagli algoritmi
% IDEA ridefinire i comandi vuoti?
\newif\ifAlgoNotCommented
% \AlgoNotCommentedtrue

% NOTE commenti
\ifAlgoPrinted
\newcommand\commentFont[1]{\footnotesize\ttfamily\textcolor{lightgray}{#1}}
\else
\newcommand\commentFont[1]{\footnotesize\ttfamily\textcolor{blue}{#1}}
\fi

% https://tex.stackexchange.com/questions/210161/
% NOTE cabiare lo stile dell'algoritmo nel testo
\makeatletter
\newcommand{\algorithmstyle}[1]{%
	\renewcommand{\algocf@style}{#1}%
}
\makeatother

% \begin{comment}
\usepackage{booktabs}
% NOTE se crea problemi nell'impaginazione degli algoritmi rimuovere senza pietà alcuna
% https://tex.stackexchange.com/questions/345737/
\makeatletter
\newcommand\fs@booktabsruled{%
	\def\@fs@cfont{\bfseries\strut}\let\@fs@capt\floatc@ruled
	\def\@fs@pre{\hrule height\heavyrulewidth depth0pt \kern\belowrulesep}%
	\def\@fs@mid{\kern\aboverulesep\hrule height\lightrulewidth\kern\belowrulesep}%
	\def\@fs@post{\kern\aboverulesep\hrule height\heavyrulewidth\relax}%
	\let\@fs@iftopcapt\iftrue
	% NOTE aggiunto dal codice sorgente di algorithm2e
	\let\@mathsemicolon=\;\def\;{\ifmmode\@mathsemicolon\else\@endalgoln\fi}%
}
\makeatother

% NOTE senza questo if condizionale succedono disastri
% NOTE -> i file standalone non compilano
\newif\ifstandalone
\ifstandalone\else
\usepackage{float}
\newfloat{algorithm}{h}{alg}[section]
\floatname{algorithm}{Algoritmo}

\floatstyle{booktabsruled}
\restylefloat{algorithm}
\fi
% \end{comment}

% NOTE correzione temporanea per sovrascrittura dell'ambiente 'algorithm'
% NOTE l'errore risiede nel codice soprastante
% \NewDocumentCommand{\InputAlgo}{m}{%
% 	\renewcommand{\;}{\par}% intepreta il '\;' come fine riga
% 	\FigureOfAlgotrue% prints explanatory image
% 	\input{#1}% TeX command
% 	\FigureOfAlgofalse% reset setting
% 	\renewcommand{\;}{\mskip\thickmuskip}% p.357 of TeXbook
% }

% https://tex.stackexchange.com/questions/153646/algorithm2e-disabling-line-numbers-for-specific-lines
% NOTE Disabling line numbers for specific lines with 'linesnumbered' option on
\let\oldnl\nl% Store \nl in \oldnl
\newcommand{\nonl}{%
	\renewcommand{\nl}{\let\nl\oldnl}%
}% Remove line number for one line

% \SetKwProg{keyword}{prima}{dopo}{fine}
\SetKwProg{prototype}{}{}{}
\SetKwProg{function}{Funzione}{}{}
\SetKwProg{procedure}{Procedura}{}{}

% NOTE didascalie ed Elenco degli algoritmi
\SetAlgorithmName{Algoritmo}{algoritmo}{Elenco degli algoritmi}
\SetAlgoProcName{Procedura}{procedura}
\SetAlgoFuncName{Funzione}{funzione}
% \SetAlgorithmName{}{Algoritmo}{Elenco degli algoritmi}
% \SetAlgoProcName{}{Procedura}
% \SetAlgoFuncName{}{Funzione}

% NOTE spaziatura nel testo
% default: smallskip
\SetAlgoSkip{smallskip}
% \SetAlgoSkip{medskip}
% \SetAlgoSkip{bigskip}

% NOTE indentazione
% \SetInd{0.5em}{1em}% <- stile compatto
\SetInd{0.25em}{2em}

% NOTE didascalia
\SetAlCapSkip{15ex}
% \SetAlgoCaptionSeparator{ --}
\SetAlgoCaptionSeparator{:}
% \SetAlgoCaptionSeparator{}
% \SetAlgoRefName{}

% NOTE stile etichette
\SetNlSty{texttt}{(}{)}
% \SetAlgoNlRelativeSize{0}
\SetNlSkip{.5em}

% NOTE definizioni di inizio e fine blocco
% \SetStartEndCondition{ (}{) }{}% c-like
\SetStartEndCondition{ }{ }{}
% \AlgoDisplayBlockMarkers\SetAlgoBlockMarkers{\{}{\}}%
% \AlgoDisplayBlockMarkers\SetAlgoBlockMarkers{begin}{end}%

% NOTE ridefinizione sei comandi in inglese
\SetKwInput{KwIng}{Ingresso}
\SetKwInput{KwData}{Dati}
\SetKwInput{KwUsc}{Uscita}
\SetKwInput{KwResult}{Risultato}

% NOTE parole chiave definite direttamente in italiano
\SetKwInput{Ingresso}{Ingresso}
\SetKwInput{Dati}{Dati}
\SetKwInput{Uscita}{Uscita}
\SetKwInput{Risultato}{Risultato}
\SetKwInput{DataStructures}{\footnotesize{\fbox{Strutture dati}}}

% NOTE ridefinizione comandi italiani

% NOTE abilita opzione 'longend', 'shortend' è dafault
% \SetKwIF{Sea}{AltSe}{Altrimenti}
%     	{se}{allora}{altrimenti se}{altrimenti}{fine\ se}% <- mancava 'se'
%
% \SetKwFor{Finche}{finché}{fai}{fine\ finché}% <- mancava 'finché'

\SetKwInOut{Input}{\small{Input}}
\SetKwInOut{Output}{\small{Output}}
% NOTE ridefinizione comandi per dispense LFC
% \SetKwInOut{Input}{\footnotesize{\fbox{input}}}
% \SetKwInOut{Output}{\footnotesize{\fbox{output}}}

% WARNING comandi già specificati nei pacchetti matematici
% \NewDocumentCommand\Let{mm}{#1 \textleftarrow #2}
% \newcommand\Tau{\mathcal{T}}

% NOTE Linguaggi Formali e Compilatori
\SetKw{Break}{break}
\SetKw{Continue}{continue}
\SetKw{Init}{init}
\SetKw{Push}{push}
\SetKw{Pop}{pop}
\SetKw{Set}{set}
\SetKw{Marca}{marca}
\SetKw{Flag}{flag}
\SetKw{Add}{add}

% \SetKwProg{Fn}{}{\string:}{}
\SetKwFunction{Closure}{closure}

\usepackage{xspace}
\providecommand{\myyes}{\textcolor{ForestGreen}{\textbf{yes}}\xspace}
\providecommand{\myno}{\textcolor{red}{\textbf{no}}\xspace}

\SetKw{Print}{stampa}

% NOTE cicli
% WARNING fa attenzione lasca degli spazi dalla definizione delle keyword, pena errori incorreggibili
% \SetKwFor{From}{da}{fai}
\SetKwFor{From}{from}{do}

% \SetKw{DownTo}{fino a}
\SetKw{DownTo}{until}

% \SetKw{Step}{con passo}
\SetKw{Step}{with step}

% NOTE logica booleana - boolean logic
\SetKw{True}{true}
\SetKw{False}{false}
\SetKw{And}{and}
\SetKw{Or}{or}
\SetKw{Not}{not}
\SetKw{To}{a}

% NOTE Algoritmi e Strutture Dati

% NOTE utilizzato nel testo
\newcommand\alert[1]{\textcolor{Blue}{#1}}
% NOTE mi assicura la compatibilità quando copio testo dalle slide
\newcommand\blink[1]{\textcolor{rose}{#1}}

% NOTE quando ometto del codice scritto in precedenza
\newcommand\omitted{\([\ldots]\)}

% NOTE definizione stili
\SetFuncSty{textsf}
\SetArgSty{upright}
\SetDataSty{textsc}
% \SetKwSty{}

% NOTE tipi
\SetKw{Int}{int}
\SetKw{Real}{float}
\SetKw{Bool}{bool}

% NOTE Strutture Dati
\SetKwData{fifo}{fifo}
\SetKwData{lifo}{lifo}

% NOTE funzioni ausiliarie
% TODO ho usato maxFunction, da cambiare
\SetKwFunction{MathMax}{max}
\SetKwFunction{MathMin}{min}
% NOTE vedi comandi
\SetKwFunction{iif}{iif}

\SetKwFunction{Swap}{swap}
% NOTE dove l'ho utilizzato?
\newcommand\sSwap[2]{\ensuremath{#1 \leftrightarrow #2}}

% NB Font
% \newcommand{\fontproc}[1]{\textsf{{\small #1}}}
\newcommand{\fonttype}[1]{\textsc{#1}\xspace}
\newcommand{\fontvar}[1]{\textit{#1}\xspace}

% NOTE variabili in comune
\newcommand\Max{\fontvar{max}} % variables, use \max for the math function
\newcommand\Min{\fontvar{min}} % variables, use \min for the math function
\newcommand\In{\fontvar{in}}
\newcommand\Out{\fontvar{out}}
\newcommand\Temp{\fontvar{temp}}
\newcommand\Item{\fonttype{Item}}
\newcommand{\priority}{\fontvar{priority}}

% NOTE valori
\SetKw{This}{this}
\SetKw{Self}{self}
\SetKw{Delete}{delete}
\SetKw{Nil}{nil}
\SetKw{Null}{null}
\SetKw{Choice}{scelta}
\SetKw{Sucess}{successo}
\SetKw{Failure}{fallimento}

% NOTE keyword
\SetKwInput{precondition}{precondition}
\SetKwFunction{new}{new}
\SetKwFunction{delete}{delete}

% NOTE stile definito all'inizio
\SetCommentSty{commentFont}
% \newcommand\mycommfont[1]{\rmfamily\textcolor{blue}{#1}}
% \SetCommentSty{mycommfont}
% NOTE \SetKwComment{<cmd>}{<before>}{<after>}
\SetKwComment{Comment}{//~}{}
% NOTE serve per l'introduzione, dove si calcola la complessità
\SetKwComment{Rem}{}{}
% \newcommand{\REMF}[1]{\Comment*[f]{#1}}
% \newcommand{\REMR}[1]{\Comment*[r]{#1}}

% NOTE definizione di una keyword di tipo vettore dove:
% #1 - tipo di dato contenuto nell'array
% #2 - valore 0-esimo o estremo sinisto
% #3 - valore 0-esimo o estremo destro
% OPTIMIZE: \SetKwArray{Array}{array}
% TODO non utilizzare 'g' come parametro, sostituirlo con 'o'
\NewDocumentCommand\Array{moo}{%
	\ensuremath{%
	\KwSty{#1}% <- spazio
		\IfNoValueTF{#2}{%
			[\,]% <- inizialmente vuoto
		}{%
		\IfNoValueTF{#3}{%
			[#2]% <- valore iniziale
		}{%
			[#2\dots#3]% <- intervallo
		}}}%
	\xspace%
}

% NOTE definizione di una keyword di tipo matrice dove:
% #1: tipo di dato contenuto nella matrice
% #2: se non viene specificato viene stampata la matrice '#1[][]'
% #3: se viene specificato viene stampata la matrice '#1[#2][#3]'
\NewDocumentCommand\Matrix{mO{}O{}}{%
	\ensuremath{%
	\KwSty{#1}% <- spazio
		\IfNoValueTF{#2}{%
			[\,][\,]% <- inizialmente vuoto
		}{%
		\IfValueT{#3}{%
			[#2][#3]% <- valori iniziali
		}}}%
	\xspace%
}

\DeclareMathOperator*\Equal{==}
\newcommand\Assign{\ensuremath{\gets}\xspace}% =
\DeclareMathOperator*\Neq{\neq}% !=

% TODO migliorare la resa tipografica, seimboli troppo vicini
\newcommand\Increment[1]{%
	\ensuremath{#1}\texttt{++}%
}
\newcommand\Decrement[1]{%
	\ensuremath{#1}\texttt{++}%
}
\newcommand\Multiply[2]{%
	\ensuremath{#1 *= #2}%
}
% https://tex.stackexchange.com/questions/67912/
\newcommand\AddTo[2]{%
	\ensuremath{#1\, + \mkern-6mu = #2}%
}
\newcommand\RemoveFrom[2]{%
	\ensuremath{#1\, -= #2}%
}

% TODO da rimuovere, non semantico
% \colorlet{algAccentColor}{blue}
% \newcommand\accentColor[1]{\textcolor{algAccentColor}{#1}}

% NOTE per rappresentazione computazioni costi nell'Introduzione
\NewDocumentCommand{\costo}{ s m O{} O{} }{%
	\IfBooleanTF{#1}{%
		% NOTE comando per la riga di intestazione
		\Rem*[f]{%
			\makebox[15mm][c]{#2}% costo
			\makebox[15mm][c]{#3}% # Volte / caso migliore
			\makebox[15mm][c]{#4}% #         caso pessimo
		}%
	}{%
		% NOTE comando per descrivere i costi
		\Rem*[f]{%
			\makebox[15mm][c]{\(#2\)}% costo
			\makebox[15mm][c]{\(#3\)}% caso migliore
			\makebox[15mm][c]{#4}%     caso pessimo
		}%
	}%
}

% HACK: Change it when you use "book" as document class
% \renewcommand{\cftchapafterpnum}{\vspace{10pt}}
\renewcommand{\cftsecafterpnum}{\vspace{10pt}}

% \let\Chapter\chapter
% \def\chapter{\addtocontents{lol}{\protect\addvspace{10pt}}\Chapter}
\let\Section\section%
\def\section{\addtocontents{lol}{\protect\addvspace{10pt}}\Section}

% NOTE: La numerazione delle sezione si azzera all'inizio di una nuova parte
\counterwithin*{section}{part}

% IDEA: definizione nuovi comandi o ambienti
% NOTE: "chngcntr" - defines the "\counterwithin" command, useful with parts of a document
% NOTE: "comment" - fornisce l'ambiente dei commenti
% NOTE: "etoolbox" - contiene il costrutto if-then-else più altri strumenti utili
% NOTE: "pgffor" - fornisce il costrutto "foreach"
% NOTE: "textcomp" - definisce la macro "textquotesingle" e formatta i numeri
% NOTE: "xargs" - use more than one optional parameter in a new commands
\usepackage{
	chngcntr,
	comment,
	etoolbox,
	pgffor,
	xargs,
	xstring,
	calc
}

% NOTE dipendenza 'xcolor'
% \colorlet{darker}{blue!50!black}

% WARNING it produces a clashing error with the tocloft package if the TOC is displayed
% NB bisogna caricarlo per ultimo
% NOTE gestisce i link testuali all'interno del documento
\usepackage{hyperref}
\hypersetup{
	,colorlinks	= true
	% ,allcolors = darker
	,linkcolor 	= red
	,anchorcolor = black
	,citecolor 	= green
	,filecolor 	= cyan
	,menucolor 	= red
	,runcolor 	= cyan
	,urlcolor 	= magenta
	,pdfauthor	= {Nardi, Emanuele}
	% pdfpagemode=FullScreen
}

% NOTE specifica la mail con link
\newcommand{\mail}[1]{\href{mailto:#1}{\texttt{#1}}}

% WARNING deve essere caricato dopo il pacchetto 'amsmath'
% NOTE non funziona come dovrebbe, leggere la documentazione
% \usepackage{cleveref}

% IDEA: compilare indipendente i file
% NOTE: "subfiles" - TODO
% NOTE: "standalone" - TODO
% NOTE: "docmute" - TODO
% WARNING: NON CAMBIARE L'ORDINE DI CARICAMENTO DEI PACCHETTI
\usepackage[
	,subpreambles	= true
	,mode			= buildnew
]{standalone}
\usepackage{subfiles}
% \usepackage{docmute}

% TODO: implementare in tutti i file
% % NOTE 'bookmark' - crea e gestisce i segnalibri
% NB i segnalibri vengono aperti fino al 1° livello
\usepackage[
	,open
	,openlevel = 1
]{bookmark}

\hypersetup{
	pdfinfo = {
		Title = {\Title},
		Author = {\Author},
		Subject = {\Subject},
		Creator = {pdflatex},
		Producer = {LaTeX},
		Keywords = {\Keywords},
		colorlinks = true,
		backref = true,
	}
}

% OPTIMIZE: non utilizzare, crea overhead
% % HACK: permette di non avere warning dal package "todonotes"
\setlength{\marginparwidth}{2cm}

% NOTE: gestione dei "to-do"
% OPTIMIZE: da rimuovere, genera overhead in fase di compilazione
\usepackage[
	,colorinlistoftodos
	,prependcaption
	,textsize = tiny
]{todonotes}

% NOTE: Definizione nuovi comandi per il pacchetto "notes"
\newcommandx{\insicuro}[2][1=]{\todo[linecolor=red,backgroundcolor=red!25,bordercolor=red,#1]{#2}}
\newcommandx{\cambia}[2][1=]{\todo[linecolor=blue,backgroundcolor=blue!25,bordercolor=blue,#1]{#2}}
\newcommandx{\info}[2][1=]{\todo[linecolor=OliveGreen,backgroundcolor=OliveGreen!25,bordercolor=OliveGreen,#1]{#2}}
\newcommandx{\miglioramento}[2][1=]{\todo[linecolor=Plum,backgroundcolor=Plum!25,bordercolor=Plum,#1]{#2}}
\newcommandx{\nascosto}[2][1=]{\todo[disable,#1]{#2}}

%
% NOTE: Permette di inserisce una pagina bianca
\newcommand{\blankpage}{%
	\null%
	\thispagestyle{empty}%
	\addtocounter{page}{-1}%
	\newpage%
}

% NOTE: Simbolo per riferimento a materiali esterni
\newcommand{\ExternalLink}{
	\tikz[x = 1.2ex, y = 1.2ex, baseline = -0.05ex]{
		\begin{scope}[x = 1ex, y = 1ex]
			\clip (-0.1,-0.1) --++ (-0, 1.2) --++ (0.6, 0) --++ (0, -0.6) --++ (0.6, 0) --++ (0, -1);
			\path[draw, line width = 0.5, rounded corners = 0.5] (0,0) rectangle (1,1);
		\end{scope}
		\path[draw, line width = 0.5] (0.5, 0.5) -- (1, 1);
		\path[draw, line width = 0.5] (0.6, 1) -- (1, 1) -- (1, 0.6);
	}
}

\newcommand{\mail}[1]{\href{mailto:#1}{\texttt{#1}\ExternalLink}}

\newcommand{\mysection}[2]{\section[#1]{#1\\[.5ex]\normalsize\textit{#2}}}
\newcommand{\mysubsection}[2]{\subsection[#1]{#1\\[.5ex]\normalsize\textit{#2}}}

\newcommand{\omissis}{[\textellipsis\unkern]}

% NOTE: Linea spessa sul testo sottostante
\newcommand{\soutthick}[1]{
	\renewcommand{\ULthickness}{1.0pt}	% 2.4
		\sout{#1}%
	\renewcommand{\ULthickness}{.4pt}	% Resetting to ulem default
}

% ---------------------------------------------------------------------------- %
% ---------------------------------------------------------------------------- %
% ---------------------------------------------------------------------------- %
