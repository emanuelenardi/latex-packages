% ---------------------------------------------------------------------------- %
% ---------------------------------------------------------------------------- %
% ---------------------------------------------------------------------------- %
% Welcome to hell!
%
% If you are editing this, look. I'm sorry.
% I did not expect you to come here.
% What are doing in this barren wasteland anyway?
% Well, I guess if you've made it this far, you've grown to understand my ugly code.
% But since this is (probably) the worst part, feel free to ask me for help:
%
% emanuele.nardi@studenti.unitn.it
%
% Note to future self:
% This message does not apply to you. Enjoy hell.
% ---------------------------------------------------------------------------- %
% ---------------------------------------------------------------------------- %
% ---------------------------------------------------------------------------- %

% NOTE: "draft" makes LaTeX indicate hyphenation and justification problems with a small square in the right-hand margin of the problem line so they can be located quickly by a human. It also suppresses the inclusion of images and shows only a frame where they would normally occur;
% NOTE: "fleqn" typesets displayed formulas left-aligned instead of centered;
% NOTE: "leqno" places the numbering of formulas on the left hand side instead of the right;
% NOTE: "a4paper" defines the paper size;
% NOTE: "11pt" dets the size of the main font in the document. If no option is specified, 10pt is assumed.
% NOTE: specifies whether a new page should be started after the document title or not. The article class does not start a new page by default, while report and book do.
\documentclass[
	,draft
	% ,gray
	% ,fleqn
	,leqno
	,a4paper
	,11pt
	,titlepage
]{article}

% ---------------------------------------------------------------------------- %
% ---------------------------------------------------------------------------- %

% NOTE: The cmap package is intended to make the PDF files generated by pdflatex
% "searchable and copyable" in acrobat reader and other compliant PDF viewers.
\usepackage{cmap}
% \usepackage[noTeX]{mmap}       % cmap + mathematics (Unicode)

\ifxetex
	\usepackage[no-math]{fontspec}

	\usepackage{polyglossia}
	\setmainlanguage{italian}

	\usepackage{xltxtra}
	\newcommand{\tex}{\TeX}
	\newcommand{\latex}{\LaTeX}
	\newcommand{\xelatex}{\XeLaTeX}

	% TODO: specifica font predefiniti con xelatex
\else
	\usepackage[T1]{fontenc}
	\usepackage[utf8]{inputenc}
	% NOTE: languages used in the document
	% NB: l'ultima dev'essere la lingua principale del documento
	\usepackage[english, main=italian]{babel}

	% NOTE: introdotto dal progetto calliope
	% \usepackage[default, osfigures, scale=0.95]{opensans}
	\usepackage[scaled=.8]{sourcecodepro}

	% IDEA: "microtype" - migliora il riempimento delle righe
	% NOTE: activate = {true,nocompatibility} - activate protrusion and expansion
	% NOTE: final - enable microtype; use "draft" to disable
	% NOTE: tracking = true, kerning=true, spacing=true - activate these techniques
	% NOTE: factor = 1100 - add 10% to the protrusion amount (default is 1000)
	% NOTE: stretch = 10, shrink = 10 - reduce stretchability/shrinkability (default is 20/20)
	\usepackage[
		% ,activate	= {true,nocompatibility}
		,activate	= {true}
		,final
		% ,draft
		,tracking	= true
		,kerning	= true
		,spacing	= true
		,factor		= 1100
		,stretch	= 10
		,shrink		= 10
	]{microtype}
\fi

% ---------------------------------------------------------------------------- %
% ---------------------------------------------------------------------------- %

\usepackage{cleveref}

% IDEA: compilare indipendente i file
% NOTE: "subfiles" - TODO
% NOTE: "standalone" - TODO
\usepackage{subfiles}
\usepackage{standalone}

% NOTE: "pifont" - introduce i simboli \cmark e \xmark
\usepackage{pifont}

% NOTE: "datetime2" - gestione e stampa delle date in vari formati
\usepackage[useregional,showdow]{datetime2}

\newcommand{\lezione}[1]{{\bfseries Lezione del \DTMdate{#1}}}

% NOTE: "mdframed" - TODO
\usepackage[framemethod = tikz]{mdframed}

% NOTE: "ulem" - permette di avere diversi tipi di sottolineatura
% NB: "normalem" - replaces underlining with italics in text emphasized by \emph
\usepackage[normalem]{ulem}

% NOTE: Linea spessa sul testo sottostante
\newcommand{\soutthick}[1]{
	\renewcommand{\ULthickness}{1.0pt}	% 2.4
		\sout{#1}%
	\renewcommand{\ULthickness}{.4pt}	% Resetting to ulem default
}

% NOTE: "tocloft" - gestione dei Table Of Content
% NB: "titles" - evita errori di caricamento dei pacchetti
\usepackage[titles]{tocloft}


% WARNING: package deprecatig
% OPTIMIZE: footmisc -> TODO
% OPTIMIZE: subfig -> subcaption
% OPTIMIZE: caption -> subcaption

\input{set-colors}
% NOTE 'geometry' - gestisce i margini della pagina
\usepackage{geometry}

% NOTE 'paper' - formato carta A4
% NOTE 'margin' - specifica tutti i margini
% NOTE 'heightrounded' - avoiding cases of "underful vbox"
% https://tex.stackexchange.com/questions/123291/
\geometry{
	% ,margin	= 2cm
	,top	= 1.5cm
	,bottom	= 2cm
	,left	= 2cm
	,right	= 2cm
	,heightrounded
}

% NOTE: gestisce la dimensione massima delle immagini
\usepackage[export]{adjustbox}

% IDEA: figure
% NOTE: "graphicx" - inserimento figure nel docuemento
% NOTE: "float" - definisce l'opzione "H" per gli oggetti fluttuanti
% NOTE: "wrapfig" - inserimento di figure di fianco al testo
\usepackage{
	graphicx,
	float,
	subcaption,
	wrapfig,
}

% NOTE: Dichiarazione dei percorsi DI DEFAULT per le immagini
\graphicspath{
	{./assets/figures/}
	{./assets/slides/}
	{./assets/icons/}
}

% WARNING: Mantenee l'ordine dei pacchetti è fonadamentale per non rompere la build
% WARNING: "semantic" da caricare sopo "amsmath"
% IDEA: math suymbols
% NOTE: "amsthm" - teoremi e dimostrazioni
% NOTE: "amsfonts" - nomi insiemi numerici
% NOTE: "amssymb" - leqslant & geqslant
% NOTE: "mathtools" - mathtools = amsmath + other stuff
% NOTE: "MnSymbol" - fornisce le freccie che utilizzo per andare a capo riga nelle liste di codice
% NOTE: "abraces" - angle brakets fine tuning
% NOTE: "braket" - permette l'uso di parentesi angolari
% NOTE: "nicefrac" - divisione in linea
% NOTE: "textgreek" - caratteri greci
% NOTE: "siunitx" - unità di misura del SI
\usepackage{
	,amsthm
	,amsfonts
	,amssymb
	,MnSymbol
	,mathtools
	% ,braket
	,semantic
	,empheq
	,nicefrac
}


% NOTE: comando definito dal package "amsmath"
\DeclarePairedDelimiter\norm{\lVert}{\rVert}
\DeclarePairedDelimiter\abs{\lvert}{\rvert}
\DeclarePairedDelimiter\Abs{\bigg\lvert}{\bigg\rvert}
\DeclarePairedDelimiter\Bracket{\lbrack}{\rbrack}

% NOTE: colorare simboli matematici
% tex.stackexchange.com/questions/21598/
\makeatletter
\def\mathcolor#1#{\@mathcolor{#1}}
\def\@mathcolor#1#2#3{%
	\protect\leavevmode
	\begingroup
		\color#1{#2}#3%
	\endgroup
}
\makeatother


% NOTE: Linguaggi formali e compilatori
\newcommand\produce{\longrightarrow}
\newcommand\deriva{\Rightarrow}

\usepackage{stackrel}

% TODO: simbolo spostato sulla destra
\newcommand\derivamultiplo{\stackrel{\ensuremath{+}}{\deriva}}
\newcommand\derivanumero[1]{\stackrel{\ensuremath{#1}}{\deriva}}

% NOTE: analisi
\renewcommand\restriction{\mathord{\upharpoonright}}

% NOTE: "uq" sta per "upquote"
\newcommand\uq{\ensuremath \text{\textquotesingle}}

% NOTE: abbreviazioni
\newcommand\ob{\overbrace}
\newcommand\ub{\underbrace}
\newcommand\us{\underset}
\newcommand\ul{\underline}
\newcommand\tn{\tn}
\newcommand\fns{\footnotesize}

% NOTE: simbolo di fine dimostrazione
\renewcommand\qedsymbol{\( \blacksquare \)}

% NOTE: teoremi, lemmi e nota bene
\newtheorem{theorem}{Teorema}
\newtheorem{lemma}[theorem]{Lemma}
\newtheorem{definition}{Definizione}
\newtheorem{corollario}{Corollario}
\newtheorem{example}{Esempio}
\newtheorem{proposition}{Proposizione}

\newtheorem*{remark}{Ricorda}
\newtheorem*{note}{Nota}
\newtheorem*{observation}{Osservazione}
\newtheorem*{hint}{Suggerimento}

% NOTE: "algorithm2e" - specifica degli algoritmi
% NOTE: "algpseudocode" - speudocodice
% NOTE: "alltt" - ridefinisce l'ambiente "verbatim"
\usepackage{
	algorithmic,
	algpseudocode,
}
% NOTE: from https://www.thomasdenney.co.uk/blog/2017/4/18/typesetting-algorithms-with-latex/
\algtext*{EndWhile}
\algtext*{EndFor}
\algtext*{EndIf}
\algtext*{EndFunction}

\algnewcommand{\SIf}[1]{\State\algorithmicif\ #1\ \algorithmicthen}
\algnewcommand{\SElseIf}[1]{\State\algorithmicelse\ \algorithmicif\ #1\ \algorithmicthen}
\algnewcommand{\SElse}{\State\algorithmicelse\ }
\algnewcommand{\SWhile}[1]{\State\algorithmicwhile\ #1\ \algorithmicdo}
\algnewcommand{\SFor}[1]{\State\algorithmicfor\ #1\ \algorithmicdo}
\algnewcommand{\SForAll}[1]{\State\algorithmicforall\ #1\ \algorithmicdo}

% IDEA: tabelle
% NOTE: "array" - permette di creare delle colonne personalizzate
% NOTE: "bigstrut" -
% NOTE: "booktabs" - genera filetti professionali per le tabelle
% NOTE: "colortbl" - righe e celle colorate
% NOTE: "diagbox" - diagonal rule on a cell
% NOTE: "ltablex" - crea tabelle dalla larghezza dinamica su più pagine
% NOTE: "makecell" -
% NOTE: "multirow" - tabelle con righe multilinea
% NOTE: "tabularx" - crea tabelle dalla larghezza dinamica
\usepackage{
	array,
	tabu,
	xcolor,
	bigstrut,
	booktabs,
	colortbl,
	diagbox,
	ltablex,
	makecell,
	multirow,
	tabularx,
}
\setlength\lightrulewidth{0.1pt}

% IDEA: rotazione
% NOTE: "pdfpages" - inserimento di pdf all'interno del documento
% NOTE: "rotating" - rotazione tabelle
\usepackage{
	pdflscape,
	rotating,
}

% IDEA gestione delle pagine

% NOTE 'pdfpages' inserimento di pdf all'interno del documento
% NOTE 'afterpage' esegue il comando dopo la prossima interruzione di pagine
\usepackage{
	,pdfpages
	,afterpage
}

% NOTE permette di inserire una pagina bianca
\newcommand{\blankpage}{%
	\null%
	\thispagestyle{empty}%
	\addtocounter{page}{-1}%
	\newpage%
}

% NOTE toglie i numeri a piè pagina dalle pagine bianche
\usepackage{emptypage}

% NOTE 'fancyhdr' permette la personalizzazione di testatina e piè di pagina
% NOTE 'lastpage' riferimento all'ultima pagina
\usepackage{fancyhdr, lastpage}

\input{set-margins}
% IDEA Disegare grafici
% NOTE 'tikz' - pacchetto completo per disegnare su LaTeX
\usepackage{tikz}

% NOTE sottolibrerie del pacchetto tikz
\usetikzlibrary{
	,arrows
	,calc
	,intersections
	,matrix
	,positioning
	,shapes.geometric
	,tikzmark
	,trees
	,decorations.text
	,decorations.pathmorphing
	,decorations.pathreplacing
}

\usepackage{set-tikz-macros}

% NOTE: optional
% tex.stackexchange.com/questions/177164/
% tex.stackexchange.com/questions/341656/
\usetikzlibrary[external]
% \usepackage{pgfplots}
% \usepgfplotslibrary[
% 	external
% ]

% Macro holding the externalized sub-directory
\newcommand{\externaldirectory}{_tikz-cache/}

% NOTE: opzioni per la pre-compilazioni delle immagini create con tikz
\tikzexternalize[
	mode = graphics if exists,
	% figure list = true,
	% All externalized graphics go to the \externaldirectory
	prefix = \externaldirectory
]
% Externalise only on-demand.
\tikzexternaldisable

% IDEA: definizione nuovi comandi o ambienti
% NOTE: "chngcntr" - defines the "\counterwithin" command, useful with parts of a document
% NOTE: "comment" - fornisce l'ambiente dei commenti
% NOTE: "etoolbox" - contiene il costrutto if-then-else più altri strumenti utili
% NOTE: "pgffor" - fornisce il costrutto "foreach"
% NOTE: "textcomp" - definisce la macro "textquotesingle" e formatta i numeri
% NOTE: "xargs" - use more than one optional parameter in a new commands
\usepackage{
	chngcntr,
	comment,
	etoolbox,
	pgffor,
	xargs,
	xstring,
	calc
}

% NOTE dipendenza 'xcolor'
% \colorlet{darker}{blue!50!black}

% WARNING it produces a clashing error with the tocloft package if the TOC is displayed
% NB bisogna caricarlo per ultimo
% NOTE gestisce i link testuali all'interno del documento
\usepackage{hyperref}
\hypersetup{
	,colorlinks	= true
	% ,allcolors = darker
	,linkcolor 	= red
	,anchorcolor = black
	,citecolor 	= green
	,filecolor 	= cyan
	,menucolor 	= red
	,runcolor 	= cyan
	,urlcolor 	= magenta
	,pdfauthor	= {Nardi, Emanuele}
	% pdfpagemode=FullScreen
}

% NOTE specifica la mail con link
\newcommand{\mail}[1]{\href{mailto:#1}{\texttt{#1}}}

% WARNING deve essere caricato dopo il pacchetto 'amsmath'
% NOTE non funziona come dovrebbe, leggere la documentazione
% \usepackage{cleveref}


\newif\ifmaindoc
\maindocfalse

\ifmaindoc
	% WARNING: può creare probemi a tempo di compilazione
\hypersetup{
	pdfinfo = {
		pdffitwindow = true,	 		% window fit to page when opened
		pdfnewwindow = true,			% links in new PDF window
		pdftitle = {\pdfTitle},			% PDF's title
		pdfsubject = {\subject},		% subject of the document
		pdfkeywords = {\tags},			% list of keywords
		pdfauthor = {\authorName},		% author of the document
		pdfcreator = {\authorName},		% creator of the document
		pdfproducer = {\authorName},	% producer of the document
	}
}

\fi

\geometry{
	paper = a4paper,					% formato carta A4
	% margin = 2cm,               		% tutti i margini
	top = 1.5cm,						% margine superiore
	bottom = 2cm,						% margine inferiore
	left = 2cm,							% margine sinistro
	right = 2cm,						% margine destro
	heightrounded,
}


% HACK: Commenta quando usi la documentclass "exam"
\fancypagestyle{footer}{
	\fancyhf{}
	\lfoot{\authorName}
	\cfoot{\footnotesize Pagina~\thepage\ di~\pageref{LastPage}}
	\rfoot{\today}
	\renewcommand{\headrulewidth}{0pt}
	\renewcommand{\footrulewidth}{2pt} % 0.8pt
}

% NOTE: Dichiarazione dei percorsi DI DEFAULT per le immagini
\graphicspath{
	{./assets/figures/}
	{./assets/slides/}
	{./assets/icons/}
}

\newcommand{\lstinputpath}[1]{\lstset{inputpath=#1}}

% NOTE: Dichiarazione dei percorsi DI DEFAULT per il codice
\lstinputpath{./assets/codes/}

% NOTE: Permette di inserisce una pagina bianca
\newcommand{\blankpage}{%
	\null%
	\thispagestyle{empty}%
	\addtocounter{page}{-1}%
	\newpage%
}

% NOTE: Simbolo per riferimento a materiali esterni
\newcommand{\ExternalLink}{
	\tikz[x = 1.2ex, y = 1.2ex, baseline = -0.05ex]{
		\begin{scope}[x = 1ex, y = 1ex]
			\clip (-0.1,-0.1) --++ (-0, 1.2) --++ (0.6, 0) --++ (0, -0.6) --++ (0.6, 0) --++ (0, -1);
			\path[draw, line width = 0.5, rounded corners = 0.5] (0,0) rectangle (1,1);
		\end{scope}
		\path[draw, line width = 0.5] (0.5, 0.5) -- (1, 1);
		\path[draw, line width = 0.5] (0.6, 1) -- (1, 1) -- (1, 0.6);
	}
}

\newcommand{\mail}[1]{\href{mailto:#1}{\texttt{#1}\ExternalLink}}

\newcommand{\mysection}[2]{\section[#1]{#1\\[.5ex]\normalsize\textit{#2}}}
\newcommand{\mysubsection}[2]{\subsection[#1]{#1\\[.5ex]\normalsize\textit{#2}}}

\newcommand{\omissis}{[\textellipsis\unkern]}

% NOTE: Linea spessa sul testo sottostante
\newcommand{\soutthick}[1]{
	\renewcommand{\ULthickness}{1.0pt}	% 2.4
		\sout{#1}%
	\renewcommand{\ULthickness}{.4pt}	% Resetting to ulem default
}


\input{set-listing}
\newcolumntype{C}{>{$}c<{$}}
\newcolumntype{L}{>{$}l<{$}}
\newcolumntype{R}{>{$}r<{$}}

% OPTIMIZE: ci sono modi migliori per farlo
\newcolumntype{!}{>{\global\let\currentrowstyle\relax}}
\newcolumntype{^}{>{\currentrowstyle}}
\newcommand{\rowstyle}[1]{\gdef\currentrowstyle{#1}%
	#1\ignorespaces%
}


% HACK: Change it when you use "book" as document class
% \renewcommand{\cftchapafterpnum}{\vspace{10pt}}
\renewcommand{\cftsecafterpnum}{\vspace{10pt}}

% \let\Chapter\chapter
% \def\chapter{\addtocontents{lol}{\protect\addvspace{10pt}}\Chapter}
\let\Section\section%
\def\section{\addtocontents{lol}{\protect\addvspace{10pt}}\Section}

% NOTE: La numerazione delle sezione si azzera all'inizio di una nuova parte
\counterwithin*{section}{part}

% % HACK: permette di non avere warning dal package "todonotes"
\setlength{\marginparwidth}{2cm}

% NOTE: gestione dei "to-do"
% OPTIMIZE: da rimuovere, genera overhead in fase di compilazione
\usepackage[
	,colorinlistoftodos
	,prependcaption
	,textsize = tiny
]{todonotes}

% NOTE: Definizione nuovi comandi per il pacchetto "notes"
\newcommandx{\insicuro}[2][1=]{\todo[linecolor=red,backgroundcolor=red!25,bordercolor=red,#1]{#2}}
\newcommandx{\cambia}[2][1=]{\todo[linecolor=blue,backgroundcolor=blue!25,bordercolor=blue,#1]{#2}}
\newcommandx{\info}[2][1=]{\todo[linecolor=OliveGreen,backgroundcolor=OliveGreen!25,bordercolor=OliveGreen,#1]{#2}}
\newcommandx{\miglioramento}[2][1=]{\todo[linecolor=Plum,backgroundcolor=Plum!25,bordercolor=Plum,#1]{#2}}
\newcommandx{\nascosto}[2][1=]{\todo[disable,#1]{#2}}

% NOTE: "xcolor" - permette di definire colori personalizzati
\usepackage[
	,dvipsnames
	,svgnames
	,x11names
	,table
	% ,gray
]{xcolor}


\colorlet{bodyColor}{blue!50!white}
\colorlet{driverColor}{blue}

% \colorlet{alternative}{red!50!white}

\colorlet{rose}{red!50!white}
% \colorlet{accent}{red!70}

% NOTE: dichiarato in set-math
\colorlet{brighter}{blue!50!white}
\colorlet{darker}{blue!50!black}

\newcommand\darker[1]{{\color{darker} #1}}

% NOTE: Definizione nuovi colori
% \definecolor{burgundy}{rgb}{0.5, 0.0, 0.13}
% \definecolor{cyan}{rgb}{0.0,0.6,0.6}
% \definecolor{darkblue}{rgb}{0.0,0.0,0.6}
% \definecolor{gray}{rgb}{0.4,0.4,0.4}

% NOTE: TODO
% \colorlet{accentColor}{blue}
% \colorlet{purple}{MediumPurple1}
% \colorlet{greenYellow}{Chartreuse2}

% NOTE: Abbreviazioni colori
% \newcommand{\darkblue}[1]{\textcolor{darkblue}{#1}}
% \newcommand{\cyan}[1]{\textcolor{cyan}{#1}}
% \newcommand{\gray}[1]{\textcolor{gray}{#1}}

% NOTE: "pifont" - introduce i simboli \cmark e \xmark
\usepackage{pifont}

% NOTE: Marking
\newcommand{\cmark}{\textcolor{ForestGreen}{\ding{51}}}
\newcommand{\xmark}{\textcolor{Red}{\ding{55}}}

\newcommand{\tick}{\cmark}

% OPTIMIZE: rimuovere dai documenti
% \newcommand{\blue}[1]{\textcolor{blue}{#1}}
% \newcommand{\azure}[1]{\textcolor{SkyBlue}{#1}}
% \newcommand{\green}[1]{\textcolor{ForestGreen}{#1}}
% \newcommand{\greenlight}[1]{\textcolor{green!30}{#1}}
% \newcommand{\red}[1]{\textcolor{red}{#1}}
% \newcommand{\orange}[1]{\textcolor{red!50}{#1}}
% \newcommand{\purple}[1]{\textcolor{purple}{#1}}

% NOTE toglie l'intentazione a tutto il documento
\setlength{\parindent}{0ex}
\setlength{\parskip}{1ex}

% NOTE 'enumitem' - permette di personalizzare gli elenchi puntati
% NOTE lista in linea
\usepackage[inline]{enumitem}

% NOTE imposta il trattino negli elenchi puntati come marcatore
% \renewcommand{\labelitemi}{\normalfont\bfseries\textendash}

% NOTE diminuisce GLOBALMENTE la distanza fra i punti
\setlist{
	% itemsep = 3pt, % default: 3pt
	% topsep = 3pt   % default: >3pt
}

% NOTE definizione liste personalizzate
% \newlist{<name>}{<type>}{<max-depth>}

\newlist{compactlist}{itemize}{2}
\setlist[compactlist]{label=\textbullet, noitemsep, topsep = 0pt, parsep = 0pt, partopsep = 0pt}

\newlist{semicompactlist}{itemize}{2}
\setlist[semicompactlist]{label=\textbullet, noitemsep, topsep = 0pt, parsep = 5pt, partopsep = 0pt}

% NB dipendenza 'set-tikz-macros'
\newlist{circledlist}{enumerate}{10}
\setlist[circledlist]{label=\protect\circled{\arabic*}}

\captionsetup{
	figureposition = bottom,
	tableposition = bottom,
	font = small,
	labelfont = {sf,bf},
	justification = centering,
}

% NOTE: Ridefinisce le virgolette
\DeclareQuoteStyle{italian}%
	{\textquotedblleft}
	[\textquotedblleft]
	{\textquotedblright}
		[0.05em]
	{\textquoteleft}
	[\textquoteleft]
	{\textquoteright}

% NOTE: Impsta il tipo di virgolette
\setquotestyle{italian}


% ---------------------------------------------------------------------------- %
% ---------------------------------------------------------------------------- %
% ---------------------------------------------------------------------------- %
