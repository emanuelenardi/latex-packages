% NOTE: comando definito dal package "amsmath"
\DeclarePairedDelimiter\norm{\lVert}{\rVert}
\DeclarePairedDelimiter\abs{\lvert}{\rvert}
\DeclarePairedDelimiter\Abs{\bigg\lvert}{\bigg\rvert}
\DeclarePairedDelimiter\Bracket{\lbrack}{\rbrack}

% NOTE: colorare simboli matematici
\makeatletter
\def\mathcolor#1#{\@mathcolor{#1}}
\def\@mathcolor#1#2#3{%
	\protect\leavevmode
	\begingroup
		\color#1{#2}#3%
	\endgroup
}
\makeatother


% NOTE: Linguaggi formali e compilatori
\newcommand\produce{\longrightarrow}
\newcommand\deriva{\implies}

\usepackage{stackrel}

% TODO: da ridefinire
\newcommand\derivamultiplo{ \stackrel{\ensuremath{+}}{\deriva}}
\newcommand\derivanumero[1]{\stackrel{\ensuremath{#1}}{\deriva}}

% NOTE: analisi
\renewcommand\restriction{\mathord{\upharpoonright}}

% NOTE: "uq" sta per "upquote"
\newcommand\uq{\ensuremath \text{\textquotesingle}}

% NOTE: insiemistica
\def\vuoto{\emptyset}

% NOTE: abbreviazioni
\newcommand\ob{\overbrace}
\newcommand\ub{\underbrace}
\newcommand\us{\underset}
\newcommand\ul{\underline}
\newcommand\tn{\tn}
\newcommand\fns{\footnotesize}

% NOTE: simbolo di fine dimostrazione
\renewcommand\qedsymbol{\( \blacksquare \)}

% NOTE: teoremi, lemmi e nota bene
\newtheorem{theorem}{Teorema}
\newtheorem{lemma}[theorem]{Lemma}
\newtheorem{definition}{Definizione}
\newtheorem{corollario}{Corollario}
\newtheorem{example}{Esempio}
\newtheorem{proposition}{Proposizione}

\newtheorem*{remark}{Ricorda}
\newtheorem*{note}{Nota}
\newtheorem*{observation}{Osservazione}
\newtheorem*{hint}{Suggerimento}

% NOTE: in alternativa ai comandi definiti sopra
% \newtheorem*{lemma}{Lemma}
% \newtheorem*{theorem}{Teorema}

% tex.stackexchange.com/questions/152485/
\makeatletter
\renewcommand{\xleftrightarrow}[2][]{\ext@arrow 3359\leftrightarrowfill@{#1}{#2}}
\newcommand{\xdashrightarrow}[2][]{\ext@arrow 0359\rightarrowfill@@{#1}{#2}}
\newcommand{\xdashleftarrow}[2][]{\ext@arrow 3095\leftarrowfill@@{#1}{#2}}
\newcommand{\xdashleftrightarrow}[2][]{\ext@arrow 3359\leftrightarrowfill@@{#1}{#2}}
\def\rightarrowfill@@{\arrowfill@@\relax\relbar\rightarrow}
\def\leftarrowfill@@{\arrowfill@@\leftarrow\relbar\relax}
\def\leftrightarrowfill@@{\arrowfill@@\leftarrow\relbar\rightarrow}
\def\arrowfill@@#1#2#3#4{%
	$\m@th\thickmuskip0mu\medmuskip\thickmuskip\thinmuskip\thickmuskip
		\relax#4#1
		\xleaders\hbox{$#4#2$}\hfill
		#3$%
}
\makeatother
